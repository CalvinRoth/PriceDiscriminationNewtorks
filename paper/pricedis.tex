% Created 2021-12-12 Sun 15:49
% Intended LaTeX compiler: pdflatex
\documentclass[11pt]{article}
\usepackage[utf8]{inputenc}
\usepackage[T1]{fontenc}
\usepackage{graphicx}
\usepackage{grffile}
\usepackage{longtable}
\usepackage{wrapfig}
\usepackage{rotating}
\usepackage[normalem]{ulem}
\usepackage{amsmath}
\usepackage{textcomp}
\usepackage{amssymb}
\usepackage{capt-of}
\usepackage{hyperref}
\author{Calvin Roth}
\date{\today}
\title{}
\hypersetup{
 pdfauthor={Calvin Roth},
 pdftitle={},
 pdfkeywords={},
 pdfsubject={},
 pdfcreator={Emacs 27.2 (Org mode 9.5)},
 pdflang={English}}
\begin{document}

\newcommand{\pr}{$\mathbb{Pr}$}

\newcommand{\ppr}{\mathbb{P}\mathbb{r}}

\section{Setup}
Fix G, a directed erdos-renyi graph, let $Dom_{G}$ be the set of graphs that have the same degree sequence as G. Typically, we will use H to be a member of this set. Additionally, let $p(H)$ be profit vector of H. Throughtout this analysis all parameters are fixed.
\pr
\textbf{Primary Questions}
\begin{enumerate}
  \item What is $\| H + H^{T}\|$(What norm is the right norm is a good question) \\
  \item Let q(H) be an altered price vector, where we fix the norms terms. What is the distribution of q(H) \\
  \item What is probability distribution of these price vectors in the doman H. \\
\end{enumerate}

We want to resolve the question of what is $Pr[v(H)_{i} = k]$
\begin{align*}
  \ppr [v(H)_{i} = ] &= \ppr [ \frac{a-c}{2} + \frac{a-c}{2} \frac{\rho}{\|H+H^{T}} K(H+H^{T}, \frac{\rho}{\|H+H^{T}\|}) = k ]\\
  &= \ppr [\frac{1}{\|H+H^{T}\|} K(H^{T}+H, \frac{\rho}{\|H+H^{T}}) = \underbrace{( \frac{2}{\rho(a-c)} [k - \frac{a-c}{2}] )}_{k'}]
\end{align*}

\subsection{Number of kth neighbors}
We will define two generating functions.

The first will be $g_{0}(z) = \sum_{k=0} p_{k} z^{k}$ where $p_{k}$ is the probabality of a given node having degree k. We know that $p_{k} \approx \frac{e^{-\lambda} \lambda^{k}}{k!}$ where $\lambda = (n-1)p$. Therefore $g_{0}(z) = \sum_{k=0} \frac{e^{-\lambda}\lambda^{k}}{k!} z^{k} = e^{-\lambda} e^{z\lambda}$.

The next generating function will be of the distribution of the degree size of a neighboring vertices.  We analyzed this in networks class. We will call if $g_{1} = (z)$ and the probability distribution of this neighbor $p_{k}^{(2)}$. We have
\begin{align*}
  g_{1}(z) &= \sum_{k=0} p_{k}^{(k)} z^{k} \\
       &= \sum_{k=0} \frac{k p_{k}}{E[d]} z^{k} \\
       &= \frac{1}{E[d]} \sum_{k=0} k p_{k} z^{k} \\
       &= \frac{1}{E[d]}  z D_{z} \sum_{k=0} p_{k} z^{k}\\
       &= \frac{1}{E[d]} z D_{z} (e^{-\lambda} e^{z\lambda}) \\
       &= \frac{z e^{-\lambda} \lambda e^{z\lambda}}{E[d]}
\end{align*}

To see why these are helpful, we will now derive the size of 2 distance neighbors from a node which we will call $d_{2}$.

\begin{align*}
  g_{2}(k) &= \sum p^{(2)}_{m} * z^{m} \underbrace{P_{2}(k|m)}_{\text{Distribution of 2 neighbors given starting node had degree m}} \\
           &= \sum_{m} p^{(2)}_{m} \left( \sum_{x_{1} + x_{2} + \cdots + x_{m} =k } \prod_{j} Pr[q_{x_{j}}]  \right) \\
           &= \sum_{m} p^{(2)}_{m} {(\sum q_{i} z^{i})}^{m} \\
           &= \sum_{m} p^{(2)}_{m} (d_{1}(z))^{m} \\
  &= g_{0}(g_{1}(z))
\end{align*}

To make this transition clear, consider what $(\sum_{i=0}^{\infty} q_{i} z^{i})(\sum_{i=0}^{\infty} q_{i} z^{i})$ organized by terms of z. The term of $z^{k}$ will have coefficient $(q_{0}q_{k} + q_{1}q_{k-1}+\cdots + q_{k-1}q_{1} + q_{k}q_{0} ) = \sum_{i+j=k} q_{i}q_{j}$.

For us this is interesting because $g_{0}(z)$ is completely fixed by the true graph G(or said alternatively is exactly the same for all the H graphs with the same degree sequence) but $g_{1}$ was not fixed directly by H. $d_{1}$ is a random function dependent on the graph it is based on. I'm not sure how this $g_{1}$ correlates with the first degree information which we fixed. Importantly, is the $g_{1}$ functions the same for our graphs vs all erdos-renyi graphs with the same n and p or does this also depend indirectly on the information we fixed. Ideally it is the first case and if so we would have shown that walk sizes are a function of the true graph's information applied to a random variable which if things are nice only depends on n and p.

We can continue this for higher distance walks. Now we will use $P_{3}(k|m)$ to be the probability that a node has k 3 distance given it has m 2-distance neighbors.  The three step generating function, $g_{3}(z)$, can be expressed as
\begin{align*}
  g_{3}(z) &= \sum_{k} \sum_{m} p^{2}_{m}  P_{3} (k|m) z^{k} \\
           &= \sum_{m} p^{2}_{m} \sum_{x_{1} + x_{2}+\cdots x_{m}=k} z^{k} \prod_{j}^{m} Pr[q_{x_{j}}] \\
           &= \sum_{m} p^{2}_{m} g_{1} (z)^{m} \\
  &= g_{2} ( g_{1} ( z)) = g_{0} ( g_{1} ( g_{1} ( z)))
\end{align*}

In general, the generating function representing the walks of length $\ell$ is $g_{0} ( g_{1}^{\ell-1} ( z))$


These have been very helpful and contains much of this analysis.
https://people.cs.clemson.edu/~isafro/ns14/l15.pdf
https://static.squarespace.com/static/5436e695e4b07f1e91b30155/t/5445263ee4b0d3d410795e1f/1413817918272/random-graphs-with-arbitrary-degree-distributions-and-their-applications.pdf

\section{Norms}
Currently unsure how I can bound the norms more tightly than what was done in the price discrimination paper.
\end{document}
